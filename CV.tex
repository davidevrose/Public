\documentclass[10pt]{article}

\pagestyle{empty}

\newcommand{\ABox}{
\raisebox{3pt}{\framebox[6pt]{\rule{6pt}{0pt}}}
}
%-------------------------------------------------


%%%%%%%%%%%%%%%%%%%%%%%%%%%%%%%%%%%%%%%%%%%%%%%%%%%%%%%%%%%%%%%%%%%%%%%%%
\pagestyle{plain}                                                      %%
\usepackage{mathptmx}                                                  %%
\usepackage[T1]{fontenc}                                               %%
%%%%%%%%%% EXACT 1in MARGINS %%%%%%%                                   %%
\setlength{\textwidth}{6.5in}     %%                                   %%
\setlength{\oddsidemargin}{0in}   %% (It is recommended that you       %%
\setlength{\evensidemargin}{0in}  %%  not change these parameters,     %%
\setlength{\textheight}{8.5in}    %%  at the risk of having your       %%
\setlength{\topmargin}{0in}       %%  proposal dismissed on the basis  %%
\setlength{\headheight}{0in}      %%  of incorrect formatting!!!)      %%
\setlength{\headsep}{0in}         %%                                   %%
\setlength{\footskip}{.5in}       %%                                   %%
%%%%%%%%%%%%%%%%%%%%%%%%%%%%%%%%%%%%                                   %%
\bibliographystyle{plain}                                              %%
%%%%%%%%%%%%%%%%%%%%%%%%%%%%%%%%%%%%%%%%%%%%%%%%%%%%%%%%%%%%%%%%%%%%%%%%%


%\addtolength{\textwidth}{1.5in}
%\addtolength{\evensidemargin}{-0.75in}
%\addtolength{\oddsidemargin}{-0.75in}
%%\addtolength{\textheight}{0.0in}
%\addtolength{\topmargin}{-.5in}
%%\addtolength{\botmargin{-0.75in}

\usepackage{ae}
%\fontencoding{T1}
%\fontfamily{garamond}
%\fontseries{m}
%\selectfont
%\renewcommand{\baselinestretch}{0.5}
\usepackage{palatino}
\usepackage{fullpage,hyperref}
\usepackage[normalem]{ulem}
\usepackage{amssymb}
\usepackage{multicol}
\normalsize

\begin{document}

\centerline{\scriptsize U N C  \hspace{0.03in} D E P A R T M E N T  \hspace{0.03in}    O F \hspace{0.03in} M A T H E M A T I C S \hspace{0.03in} $\bullet$  
\hspace{0.03in} P H I L L I P S  \hspace{0.03in} H A L L , \hspace{0.03in} C B \# 3 2 5 0  \hspace{0.03in}  U N C - C H  \hspace{0.03in} $\bullet$ 
\hspace{0.03in} C H A P E L \hspace{0.03in} H I L L , \hspace{0.03in} N C  \hspace{0.03in} 2 7 5 9 9 - 3 2 5 0}

\vspace{0.04in}
\centerline{\scriptsize {E - M A I L:} \hspace{0.03in} {
        \href{mailto:davidrose@unc.edu}{\bf D A V I D R O S E @ U N C . E D U}} \hspace{0.03in}
        $\bullet$ \hspace{0.03in} W E B: \hspace{0.03in} \href{http://davidev.web.unc.edu/}{\bf H T T P : // D A V I D E V . W E B . U N C . E D U}}

\bigskip
\bigskip

\noindent
{\centerline{\Huge DAVID \hspace{0.03in} E. V. \hspace{0.01in} ROSE}}
\bigskip

{\centerline \today}

\bigskip


%% ---------------------------------------------------------------------------
\noindent
{\large \sc Research interests}

\vspace{-0.1in}
\noindent
\line(1,0){450}

\smallskip


\begin{itemize} 
\item Low-dimensional topology and TQFT
\item Representation theory
\item Homological algebra and category theory

\end{itemize}
%% ---------------------------------------------------------------------------

\bigskip
%% ---------------------------------------------------------------------------
\noindent
{\large \sc Education}

\vspace{-0.1in}
\noindent
\line(1,0){450}

\smallskip

\begin{center} {
\begin{tabular}{l l}
{\bf 2012} %May 13
& {\bf Duke University,} Durham, NC \\
& {Ph.D. in Mathematics} \\
& {Advisor: Lenny Ng}\\
& {Thesis: \emph{Categorification of Quantum $\mathfrak{sl}_3$ Projectors 
	and the $\mathfrak{sl}_3$ Reshetikhin-Turaev Invariant} 
}\\ &{
%\qquad \quad \enspace
\emph{of Framed Tangles}}\\
\\
{\bf 2007} %June 21
& {\bf Christ's College, University of Cambridge,} Cambridge, UK \\
& {Certificate of Advanced Studies (Part III) with Merit} \\
& {Essay: \emph{Dirac Operators}} \\
\\
{\bf 2006} %May 14
& {\bf The College of William and Mary,} Williamsburg, VA \\ 
& {B.S. in Mathematics and Physics, \emph{Summa Cum Laude} with highest (research) honors}\\
& {Mathematics Advisor: Ilya Spitkovsky}\\
& {Mathematics Thesis: \emph{Results concerning the Aluthge transform}}\\
& {Physics Advisor: Christopher Carone}\\
& {Physics Thesis: \emph{Minimal length uncertainty and the quantum mechanics of non-commutative}
}\\ &{
%\qquad \qquad \qquad \quad
\emph{space-time}}\\
\end{tabular}
}
\end{center}
%% ---------------------------------------------------------------------------

\bigskip
%% ---------------------------------------------------------------------------
\noindent
{\large \sc Professional experience}

\vspace{-0.1in}
\noindent
\line(1,0){450}

\smallskip

%\begin{center}
{
\begin{tabular}{l l}
{\bf July 2022 -- } & {\bf University of North Carolina}, Chapel Hill, NC \\
& {Associate Professor}\\ \\

{\bf July 2016 -- June 2022} & {\bf University of North Carolina}, Chapel Hill, NC \\
& {Assistant Professor}\\ \\

{\bf August 2012 -- July 2016} & {\bf University of Southern California}, Los Angeles, CA \\
& {Busemann Assistant Professor (NTT postdoc)}\\
\\
\end{tabular}
}
%\end{center}
%% ---------------------------------------------------------------------------


%\bigskip
%%% ---------------------------------------------------------------------------
%\noindent
%{\large \sc Invited Long-term Visits}
%
%\vspace{-0.1in}
%\noindent
%\line(1,0){450}
%
%\smallskip
%
%%\begin{center}
%{
%\begin{tabular}{l l}
%{\bf Summer 2021 --} & {\bf American Institute of Mathematics}, San Jose, CA \\
%& {Invited participant and group co-leader: "Link Homology Research Community''}\\ \\
%{\bf Fall 2018} & {\bf Kavli Institute for Theoretical Physics}, Santa Barbara, CA \\
%& {Invited participant: program on ``Quantum Knot Invariants and Supersymmetric }\\ 
%&{Gauge Theories''}\\ \\
%{\bf Spring 2017} & {\bf Isaac Newton Institute}, Cambridge, UK \\
%& {Invited participant: program on ``Homology theories in low dimensional topology''}\\
%\\
%\end{tabular}
%}
%%\end{center}
%%% ---------------------------------------------------------------------------


\newpage
%\bigskip
%% ---------------------------------------------------------------------------
\noindent
{\large \sc Honors}

\vspace{-0.1in}
\noindent
\line(1,0){450}

\smallskip

\begin{itemize}

\item \textbf{G. de B. Robinson Award} of the Canadian Mathematical Society,
%(joint with L. Tatham), %for our paper \emph{On webs in quantum type $C$}), 
2024

\item \textbf{NSF CAREER Award}, 2022

\item Invited researcher and group co-leader, {\bf American Institute of Mathematics}, San Jose, CA. 
\emph{Link homology research community}, Summer and Fall 2021.

\item Invited researcher, {\bf Kavli Institute for Theoretical Physics}, Santa Barbara, CA. 
Program on \emph{Quantum Knot Invariants and Supersymmetric Gauge Theories}, Fall 2018.

\item Invited researcher, {\bf Isaac Newton Institute}, Cambridge, UK.
Program on \emph{Homology theories in low dimensional topology}, Spring 2017.

\item {\bf Bass Fellowship} for designing and teaching the undergraduate course 
\emph{Algebraic methods in knot theory} at Duke University, 
%Award Amount: \$25,240, 
Dates: 9/1/2011 -- 5/31/2012


\item {\bf L.P. and Barbara Smith award for beginning teachers} 
in the Duke University Mathematics Department, 
February, 2009
 
\item {\bf Cambridge overseas trust scholarship,} September, 2006
 
\item {\bf William and Mary Prize in Mathematics,} 
awarded to the top graduating Mathematics major, May~2006
 
\item {\bf Phi Beta Kappa,} College of William and Mary, November, 2005

\item {\bf E. G. Clark memorial scholarship,} 
awarded to the top junior in the William and Mary Department of Physics, May 2005

\item {\bf Monroe Scholar,} College of William and Mary, Fall 2002 -- Spring 2006 

\end{itemize}
%% ---------------------------------------------------------------------------


%\newpage
\bigskip
%% ---------------------------------------------------------------------------
\noindent
{\large \sc Publications and preprints}

\vspace{-0.1in}
\noindent
\line(1,0){450}

\smallskip

\noindent Authors on papers are listed alphabetically (as is convention in pure mathematical writing)
and papers are listed reverse-chronologically within section, in order of completion.
Refereed publications are those that have been published 
(or accepted for publication) as of \today.
A dagger\textsuperscript{\textdagger} denotes a collaborator who was a student during the collaboration.

\medskip

\noindent\textbf{Refereed publications:}

\begin{enumerate}

\item 
M.~Hogancamp, D.E.V.~Rose, and P.~Wedrich.
{\bf A Kirby color for Khovanov homology}.
%Matthew Hogancamp, David E. V. Rose, and Paul Wedrich, 
2022.
\emph{Journal of the European Mathematical Society (JEMS)}.
DOI 10.4171/JEMS/1589.
\href{http://arxiv.org/abs/2210.05640}{\emph{arXiv:2210.05640}}.
60~pages.

\item
M.~Hogancamp, D.E.V.~Rose, and P.~Wedrich.
{\bf A skein relation for singular Soergel bimodules}.
2021.
to appear in \emph{Selecta Mathematica}.
\href{http://arxiv.org/abs/2107.08117}{\emph{arXiv:2107.08117}}.
34~pages.

\item 
M.~Hogancamp, D.E.V.~Rose, and P.~Wedrich.
{\bf Link splitting deformation of colored Khovanov--Rozansky homology}.
%Matthew Hogancamp, David E. V. Rose, and Paul Wedrich, 
%2021.
\emph{Proceedings of the London Mathematical Society}
%Proc. Lond. Math. Soc. 
(3) 129 (2024), no. 3, Paper No. e12620, 142 pages.
%https://doi.org/10.1112/plms.12620
%\href{http://arxiv.org/abs/2107.09590}{\emph{arXiv:2107.09590}}.
%110~pages.
%142~pages.

\item 
D.E.V.~Rose and L.~Tatham\textsuperscript{\textdagger}.
{\bf On webs in quantum type $C$}.
%David E. V. Rose and Logan Tatham,
\emph{Canadian Journal of Mathematics}, 74(3), June 2022, 793--832.
% 40 pages.
%\href{http://arxiv.org/abs/2006.02491}{\texttt{arXiv:2006.02491}}.
(For this paper, we were awarded the \textbf{G. de B. Robinson Award} by the CMS.)

\item 
D.E.V.~Rose and D.~Tubbenhauer.
{\bf HOMFLYPT homology for links in handlebodies via type $A$ Soergel bimodules},
%David E. V. Rose and Daniel Tubbenhauer, 
\emph{Quantum Topology}, 12 (2021), no. 2, 373--410. %38 pp.
%\href{http://arxiv.org/abs/1908.06878}{\texttt{arXiv:1908.06878}}.

\item 
M.~Abram\textsuperscript{\textdagger}, L.~Lamberto-Egan\textsuperscript{\textdagger}, 
A.~Lauda, and D.E.V.~Rose.
{\bf Categorification of the internal braid group action for quantum groups I: 2-functoriality}. 
%Michael T. Abram, Laffite Lamberto-Egan, Aaron D. Lauda, and David E. V. Rose, 
\emph{Pacific Journal of Mathematics},
Vol. 328 (2024), No. 1, 1--75.
%75 pages
%\href{http://arxiv.org/abs/2205.04598}{\emph{arXiv:2205.04598}}.

\item 
H.~Queffelec and D.E.V.~Rose.
{\bf Sutured annular Khovanov--Rozansky homology}.
%Hoel Queffelec and David E. V. Rose, 
\emph{Transactions of the AMS}, 370 (2018), 1285--1319. %35 pp.
%\href{http://arxiv.org/abs/1506.08188}{\texttt{arXiv:1506.08188}}.

\item 
D.E.V.~Rose and P.~Wedrich.
{\bf Deformations of colored $\mathfrak{sl}_n$ link homologies via foams}.
%David E. V. Rose and Paul Wedrich, 
\emph{Geometry and Topology}, 20 (2016), no. 6, 3431--3517. %87 pp.
%\href{http://arxiv.org/abs/1501.02567}{\texttt{arXiv:1501.02567}}.

\item 
D.E.V.~Rose and D.~Tubbenhauer.
{\bf Symmetric webs, Jones--Wenzl recursions, and $q$-Howe duality}.
%David E. V. Rose and Daniel Tubbenhauer, 
\emph{International Mathematics Research Notices}, 2016 (17): 5249--5290. %42 pp.
%\href{http://arxiv.org/abs/1501.00915}{\texttt{arXiv:1501.00915}}.

\item 
H.~Queffelec and D.E.V.~Rose.
{\bf The $\mathfrak{sl}_n$ foam $2$-category: a combinatorial formulation of Khovanov--Rozansky
homology via categorical skew Howe duality}.
%Hoel Queffelec and David E. V. Rose, 
\emph{Advances in Mathematics}, 302 (2016), 1251--1339. %89 pp.
%\href{http://arxiv.org/abs/1405.5920}{\texttt{arXiv:1405.5920}}.

\item 
A.~Lauda, H.~Queffelec\textsuperscript{\textdagger}, and D.E.V.~Rose.
{\bf Khovanov homology is a skew Howe $2$-representation of categorified quantum $\mathfrak{sl}_m$}.
%Aaron D. Lauda, Hoel Queffelec, and David E. V. Rose, 
\emph{Algebraic and Geometric Topology}, 15-5 (2015), 2517--2608. %\\ %92 pp.
%\href{http://arxiv.org/abs/1212.6076}{\texttt{arXiv:1212.6076}}.

\item 
D.E.V.~Rose.
{\bf A note on the Grothendieck group of an additive category}.
%David E. V. Rose,
\emph{Bulletin of Chelyabinsk State University}, 2015. no. 3 (358). 
Mathematics. Mechanics. Informatics. Issue 17. pp. 135--139 % 4 pp.
(Proceedings of the {\it International Conference ``Quantum topology"}).
%\href{http://arxiv.org/abs/1109.2040}{\texttt{arXiv:1109.2040}}.
 
\item 
D.E.V.~Rose.
{\bf A categorification of quantum $\mathfrak{sl}_3$ projectors and the $\mathfrak{sl}_3$ 
Reshetikhin-Turaev invariant of tangles}.
%David E. V. Rose,
\emph{Quantum Topology}, 5 (2014), no. 1, pp. 1--59. %59 pp.
%\href{http://arxiv.org/abs/1109.1745}{\texttt{arXiv:1109.1745}}.
 
\item
D.E.V.~Rose and I.~Spitkovsky.
{\bf On the numerical range behavior under the generalized Aluthge transform}.
%David E. V. Rose and Ilya M. Spitkovsky,
\emph{Linear and Multilinear Algebra}, vol. 56 no. 1\&2 (January, 2008), pp. 163--177. %15 pp.
 
\item 
D.E.V.~Rose and I.~Spitkovsky.
{\bf On the stabilization of the Aluthge sequence}.
%David E. V. Rose and Ilya M. Spitkovsky,
\emph{International Journal of Information and Systems Sciences - 
Special Issue on Matrix Analysis and Applications}, 
vol. 4 no. 1 (Spring, 2008), pp. 178--189. % 12 pp.
\end{enumerate}

\noindent\textbf{Preprints:}

\begin{enumerate}
\setcounter{enumi}{15}

\item
E.~Bodish, B.~Elias, and D.E.V.~Rose.
{\bf Spin Link Homology}.
2024.
Submitted for publication.
%to Inventiones on 8/2/24
\emph{Preprint}
\href{http://arxiv.org/abs/2407.00189}{\emph{arXiv:2407.00189}}.
137~pages.

\item
M.~Hogancamp, D.E.V.~Rose, and P.~Wedrich.
{\bf Bordered invariants from Khovanov homology}.
2024.
Submitted for publication.
% to Duke on 3/3/25
% to ASENS on 4/24/25
% to JEMS on 9/3/25
\emph{Preprint}
\href{http://arxiv.org/abs/2404.06301}{\emph{arXiv:2404.06301}}.
58~pages.

\item
E.~Bodish\textsuperscript{\textdagger}, B.~Elias, D.E.V.~Rose, and L.~Tatham.
{\bf A note on the Sundaram--Stanley bijection (or, Viennot for up-down tableaux)}.
%Elijah Bodish, Ben Elias, David E. V. Rose, and Logan Tatham,
2021.
Under revision at \emph{Abhandlungen aus dem Mathematischen Seminar der Universit\:{a}t Hamburg}.
% to Abhandlungen on 6/29/24
\emph{Preprint} 
\href{http://arxiv.org/abs/2108.11528}{\emph{arXiv:2108.11528}}.
14~pages.

\item 
E.~Bodish\textsuperscript{\textdagger}, B.~Elias, D.E.V.~Rose, and L.~Tatham\textsuperscript{\textdagger}.
{\bf Type $C$ webs}.
%Elijah Bodish, Ben Elias, David E. V. Rose and Logan Tatham,
2021.
Submitted for publication.
%Duke 4/21/21
%JEMS 5/28/21
%Inventiones 9/13/21
%Crelle 3/9/22
%Compositio 7/29/22
%ASENS ??? (rejected on 10/10/23)
%AJM 11/30/23
%CAMS 4/23/25
\emph{Preprint}
\href{http://arxiv.org/abs/2103.14997}{\emph{arXiv:2103.14997}}.
45~pages.

\item 
H.~Queffelec, D.E.V.~Rose, and A.~Sartori.
{\bf Annular Evaluation and Link Homology}.
%Hoel Queffelec, David E. V. Rose, and Antonio Sartori, 
2018.
Submitted for publication.
%J Top.
\emph{Preprint}
\href{http://arxiv.org/abs/1802.04131}{\emph{arXiv:1802.04131}}.
47 pages.

\end{enumerate}

\noindent\textbf{In preparation:}

\begin{enumerate}
\setcounter{enumi}{20}



\item
M.~Hogancamp, D.E.V.~Rose, and P.~Wedrich.
{\bf A $3$-manifold invariant from Khovanov homology}.
\emph{In preparation} (arXiv ETA: 2025).
%\href{http://arxiv.org/abs/}{\emph{arXiv:}}.
%currently 7~pages.

\item 
E.~Bodish, B.~Elias, and D.E.V.~Rose.
{\bf Type $B$ webs}.
\emph{In preparation} (arXiv ETA: 2025).

%\item
%M.~Hogancamp and D.E.V.~Rose.
%{\bf $\mathfrak{sl}_n$ link homology and the Hecke category}.
%\emph{In preparation} (arXiv ETA: Summer 2024),
%%\href{http://arxiv.org/abs/}{\emph{arXiv:}}.
%currently 7~pages.

\end{enumerate}


\newpage
%\bigskip
% ---------------------------------------------------------------------------
\noindent
{\large \sc Grants}

\vspace{-0.1in}
\noindent
\line(1,0){450}

\smallskip

\noindent To date, I have been awarded \$546,000 in grant funding:

\begin{itemize}
\item {\bf NSF CAREER Grant DMS-2144463}: 
\emph{Link homology -- in type $A$ and beyond},
PI: David Rose (percent effort: 16.67\%),
Award Amount \$425,000 (direct cost: \$285,370),
Dates: 7/1/2022 -- 6/30/2027

\item {\bf Simons Collaboration Grant}: 
\emph{Research on knot invariants, representation theory, and categorification}, 
% Grant number 523992
PI: David Rose (percent effort: 0\%), 
Award Amount: \$42,000, 
Direct Cost: \$35,000,
Dates: 9/1/2017 -- 8/31/2022

\item {\bf UNC Junior Faculty Development Award},
PI: David Rose (percent effort: 8.88\%), 
Award Amount: \$10,000, 
Dates: 1/1/2019 -- 12/31/2019

\item {\bf NSA Young Investigator Grant}: 
\emph{ Dualities in higher representation theory and low-dimensional topology}, 
% Grant number H98230-17-1-0211
PI: David Rose (percent effort at UNC: 8.33\%), 
Award Amount: \$40,000, 
Direct Cost (at UNC): \$17,391
Dates: 1/21/2016 -- 12/31/2017

\item {\bf AMS Simons Travel Grant}, 
PI: David Rose,
Award Amount: \$4,000, 
Dates: 7/1/2015 -- 12/31/2015

 \item {\bf Zumberge Fund Individual Grant Award}: \emph{ Categorified quantum groups and knot homology}, 
 PI: David Rose,
 Award Amount: \$25,000, 
 Dates: 7/1/2013 -- 6/30/2014
\end{itemize}


%\newpage
\bigskip
%% ---------------------------------------------------------------------------
\noindent
{\large \sc Conference and Seminar Talks}

\vspace{-0.1in}
\noindent
\line(1,0){450}
\smallskip

%\noindent\textbf{International:}
%
%\begin{enumerate}
%
%\item Workshop on Quiver Hecke algebra and its applications to topology, 
%Nagoya, Japan, 
%\emph{Link homology via the categorified skein module of the annulus},
%July, 8, 2019
%
%\item Interactions of low-dimensional topology and ``higher'' representation theory,
%Universit\"{a}t Z\"{u}rich, Switzerland,
%\emph{$\mathfrak{gl}_n$ homologies, annular evaluation, and symmetric webs},
%September 19, 2018
%
%\item Categorification and Higher Representation Theory,
%Institute-Mittag Leffler, Djursholm, Sweden,
%\emph{$\mathfrak{gl}_n$ homologies, annular evaluation, and symmetric webs},
%July 10, 2018
%
%\item Mathematical Conference of the Americas,
%Special Session on Symmetry in Algebra, Topology, and Physics,
%Montreal, Canada,
%\emph{Annular evaluation and link homology},
%July 26, 2017
%
%\item Quantum topology and categorified representation theory, 
%Isaac Newton Institute, Cambridge University, 
%\emph{Traces, \sout{current algebras}, and link homologies}, 
%June 26, 2017
%
%\item Institut Montpell\'{e}rain Alexander Grothendieck, S\'{e}minaire Topologies, 
%\emph{Quantum knot and link invariants from the symmetric perspective},
%September 22, 2016
%
%\item Centre for quantum geometry of moduli spaces (QGM), Aarhus, Denmark,
%\emph{From the Jones polynomial to Khovanov-Rozansky homology via skew Howe duality}, 
%July 30, 2014
%
%\item Centre de recherches math\'{e}matiques workshop on 
%Categorification and geometric representation theory, Montreal,
%\emph{The $\mathfrak{sl}_n$ foam 2-category via skew Howe duality}, June 9, 2014
%
%\item Australian National University, Algebra and Topology Seminar, 
%\emph{Quantum link invariants and skew Howe duality}, March 18, 2014
%
%\item Institut de Math\'{e}matiques de Jussieu, Paris, \emph{Khovanov homology via categorified quantum groups},
%December 17, 2013
%
%\item The eighth workshop on numerical ranges and numerical radii (WONRA), Universit\"{a}t Bremen, 
%\emph{On the numerical range behavior under the generalized Aluthge transform}, July 15, 2006
%
%\end{enumerate}
%
%\noindent\textbf{Domestic:}
\begin{enumerate}
%\setcounter{enumi}{11}

\item ICERM Workshop on Webs in Algebra, Geometry, Topology and Combinatorics, 
\emph{TBD},
December 8--12, 2025 (forthcoming)

\item Categorification in Low Dimensional Topology,
Bochum, Germany,
\emph{Towards a categorification of the Turaev--Viro TQFT} (four 1.5-hour lectures),
July 22--25, 2025

\item Columbia University, 
Geometric Topology Seminar,
\emph{Towards a higher TQFT from Khovanov homology},
March 7, 2025

\item Stanford University, 
Topology Seminar,
\emph{Towards a categorification of the Turaev--Viro TQFT},
June 4, 2024

\item Uppsala University, 
Algebra Seminar,
\emph{Spin link homology},
May 21, 2024

\item UC Berkeley, 
String-Math Seminar,
\emph{Towards a categorification of the Turaev--Viro TQFT},
April 15, 2024

\item Michigan State University,
Topology Seminar,
\emph{A Kirby color (or two) for Khovanov homology},
September 19, 2023

\item Michigan State University,
RTG Seminar,
\emph{The Temperley--Lieb and Bar-Natan categories},
September 18, 2023

%\item Fall AMS Eastern Sectional Meeting 
%special session on Combinatorial and Categorical Techniques in Representation Theory, 
%University at Buffalo (SUNY), in Buffalo, NY.
%September 9-10, 2023 (declined)

\item University Quantum Symmetries Lectures (UQSL),
\emph{A Kirby color for Khovanov homology},
January 12, 2023

\item QUAntum groups, Categorification, Knot invariants, and Soergel bimodules II,
University of Oregon,
\emph{A Kirby color for Khovanov homology},
August 11, 2022

\item University of North Carolina at Chapel Hill,
Geometric Methods in Representation Theory Seminar, 
\emph{Type $C$ Webs},
April 29, 2022

\item Categorical Methods in Representation Theory and Quantum Topology,
University of Virginia,
\emph{Type $C$ Webs},
April 17, 2022

\item Triangle Area Graduate Mathematics Conference (plenary faculty lecture), 
North Carolina State University,
\emph{Quantum knot invariants and webs},
April 2, 2022

\item North Carolina State University, Algebra and Combinatorics Seminar, 
\emph{Type $C$ Webs},
March 28, 2022

\item Penn State University, Colloquium, 
\emph{Link Homology},
March 21, 2022

\item American Institute of Mathematics, Fundamentals of Link Homology Seminar,
\emph{Triply-graded link homology, Soergel bimodules, (and Traces)},
August 4, 2021

\item American Institute of Mathematics, Fundamentals of Link Homology Seminar,
\emph{Basics of Khovanov homology},
June 30, 2021

\item Spring AMS Western Sectional Meeting 
special session on Diagrammatic and Combinatorial Methods in Representation Theory, 
\emph{Type $C$ Webs},
May 1, 2021

\item UC Davis, Algebra \& Discrete Mathematics Seminar,
\emph{Type $C$ Webs},
March 11, 2021

\item Fall AMS Western Sectional Meeting 
special session on Monoidal Categories in Representation Theory, 
University of Utah, Salt Lake City, 
\emph{Webs in Type $C$},
October 25, 2020

\item QUAntum groups, Categorification, Knot invariants, and Soergel bimodules,
University of Oregon,
\emph{Webs in Type $C$},
August 12, 2020

\item Spring AMS Southeastern Sectional Meeting 
special session on Categorical Representation Theory and Beyond, 
University of Virginia,
\emph{On the quantum type C spider},
March 15, 2020 (Cancelled due to COVID-19 pandemic)

\item ICERM Workshop on Illustrating Number Theory and Algebra, 
Brown University, 
\emph{Webs, foams, knot invariants, and representation theory},
October 21, 2019

%%%%1
\item Workshop on Quiver Hecke algebra and its applications to topology, 
Nagoya, Japan, 
\emph{Link homology via the categorified skein module of the annulus},
July, 8, 2019

\item UC Berkeley, String-Math Seminar,
\emph{$\mathfrak{gl}_n$ homologies, annular evaluation, and symmetric webs},
October 8, 2018

%%%%2
\item Interactions of low-dimensional topology and ``higher'' representation theory,
Universit\"{a}t Z\"{u}rich, Switzerland,
\emph{$\mathfrak{gl}_n$ homologies, annular evaluation, and symmetric webs},
September 19, 2018

\item CQ3MI workshop,
University of Southern California, 
\emph{$\mathfrak{gl}_n$ homologies, annular evaluation, and symmetric webs},
July 19, 2018

%%%%3
\item Categorification and Higher Representation Theory,
Institute-Mittag Leffler, Djursholm, Sweden,
\emph{$\mathfrak{gl}_n$ homologies, annular evaluation, and symmetric webs},
July 10, 2018

\item Quantum Knot Homology and Supersymmetric Gauge Theory, 
Aspen Center for Physics, 
\emph{Annular evaluation and link homology},
March 5, 2018

%%%%4
\item Mathematical Conference of the Americas,
Special Session on Symmetry in Algebra, Topology, and Physics,
Montreal, Canada,
\emph{Annular evaluation and link homology},
July 26, 2017

%%%%5
\item Quantum topology and categorified representation theory, 
Isaac Newton Institute, Cambridge University, 
\emph{Traces, \sout{current algebras}, and link homologies}, 
June 26, 2017

\item Triangle Area Graduate Mathematics Conference (plenary faculty lecture), 
Duke University, 
\emph{Knot invariants and dualities}, 
April 23, 2017

\item North Carolina State University, Geometry and Topology Seminar, 
\emph{Quantum knot and link invariants from the symmetric perspective},
March 1st, 2017

%%%%6
\item Institut Montpell\'{e}rain Alexander Grothendieck, S\'{e}minaire Topologies, 
\emph{Quantum knot and link invariants from the symmetric perspective},
September 22, 2016

\item University of North Carolina at Chapel Hill, 
\emph{Link homology via traces, current algebras, and dualities},
February 17, 2016

\item University of North Carolina at Chapel Hill, 
Mathematics Colloquium,
\emph{Low-dimensional topology, representation theory, and categorification},
February 16, 2016

\item Dartmouth College, Mathematics Colloquium,
\emph{Low-dimensional topology, representation theory, and categorification},
February 9, 2016

\item Joint Mathematics Meeting, AMS special session on 
Geometric and categorical methods in representation theory, 
Seattle, Washington,
\emph{Current algebras, Khovanov-Rozansky homology, and annular link invariants},
January 8, 2016

\item Joint Mathematics Meeting, AMS special session on Topological Representation Theory, 
Seattle, Washington,
\emph{Howe dualities and link invariants},
January 6, 2016

\item Knots in Washington -- Plenary Talk, George Washington University,
\emph{Khovanov-Rozansky homology and current algebras},
December 4th, 2015

\item Duke University, Geometry/Topology Seminar, 
\emph{Quantum knot invariants and Howe dualities},
April 7th, 2015

\item North Carolina State University, Geometry and Topology Seminar, 
\emph{Quantum knot invariants and Howe dualities},
April 6th, 2015

\item University of North Carolina, Physically Inspired Mathematics Seminar, 
\emph{Annular Khovanov-Rozansky homology},
April 2nd, 2015

\item University at Buffalo (SUNY), Mathematics Colloquium, 
\emph{Categorification in topology and representation theory},
December 8th, 2014

\item The Joint Los Angeles Topology Seminar, UCLA, 
\emph{Annular Khovanov homology via trace decategorification},
December 1, 2014

\item UC Davis, Algebra \& Discrete Mathematics Seminar,
\emph{Khovanov-Rozansky homology via categorified quantum groups and skew Howe duality},
November 24, 2014

\item Fall AMS Southeastern Section Meeting special session on 
Algebraic structures motivated by Knot Theory, 
University of North Carolina at Greensboro, 
\emph{Annular Khovanov homology via trace decategorification},
November 8, 2014

%%%%7
\item Centre for quantum geometry of moduli spaces (QGM), Aarhus, Denmark,
\emph{From the Jones polynomial to Khovanov-Rozansky homology via skew Howe duality}, 
July 30, 2014

\item NSF/CBMS Regional Conference on Higher Representation Theory, 
North Carolina State University, 
\emph{From the Jones polynomial to Khovanov-Rozansky homology via skew Howe duality}, 
July 7, 2014

%%%%8
\item Centre de recherches math\'{e}matiques workshop on 
Categorification and geometric representation theory, Montreal,
\emph{The $\mathfrak{sl}_n$ foam 2-category via skew Howe duality}, June 9, 2014

%%%%9
\item Australian National University, Algebra and Topology Seminar, 
\emph{Quantum link invariants and skew Howe duality}, March 18, 2014

%%%%10
\item Institut de Math\'{e}matiques de Jussieu, Paris, 
\emph{Khovanov homology via categorified quantum groups},
December 17, 2013

\item University of North Carolina, Physically Inspired Mathematics Seminar, 
\emph{Quantum link invariants and (higher) representation theory 
via skew Howe duality}, March 22, 2013

\item George Washington University, Topology Seminar, 
\emph{Quantum link invariants and (higher) representation theory 
via skew Howe duality}, March 21, 2013

\item UC Riverside, Topology Seminar, 
\emph{Quantum link invariants and (higher) representation theory 
via skew Howe duality}, March 5, 2013

\item Caltech, Geometry and Topology Seminar, 
\emph{Khovanov homology, categorified quantum groups, 
and skew-Howe duality}, January 18, 2013

\item Claremont Colleges, Topology Seminar, 
\emph{Quantum link invariants and (higher) representation theory 
via skew Howe duality}, October 30, 2012

\item UCLA, Topology Seminar, 
\emph{Foams, Khovanov Homology, and Categorical Skew Howe Duality}, 
October 10, 2012

\item University of Southern California, Geometry and Topology Seminar, 
\emph{A categorification of quantum $\mathfrak{sl}_3$ projectors 
and the $\mathfrak{sl}_3$ Reshetikhin-Turaev invariant of tangles}, April 27, 2012

\item University of Virginia, Geometry Seminar, 
\emph{A categorification of quantum $\mathfrak{sl}_3$ projectors 
and the $\mathfrak{sl}_3$ Reshetikhin-Turaev invariant of tangles}, February 7, 2012

\item Baton Rouge Young Topologists Research Retreat, Louisiana State University, 
\emph{A categorification of quantum $\mathfrak{sl}_3$ projectors 
and the $\mathfrak{sl}_3$ Reshetikhin-Turaev invariant of tangles}, January 9, 2012

\item Rice University, VIGRE Topology Seminar, 
\emph{On Bar-Natan's ``Khovanov's homology for tangles and cobordisms''}, March 9, 2010

%%%%11
\item The eighth workshop on numerical ranges and numerical radii (WONRA), 
Universit\"{a}t Bremen, 
\emph{On the numerical range behavior under the generalized Aluthge transform}, 
July 15, 2006


\end{enumerate}




\bigskip
%\newpage
% ---------------------------------------------------------------------------
\noindent
{\large \sc Advising and Teaching Activities}

\vspace{-0.1in}
\noindent
\line(1,0){450}

\smallskip
\noindent\textbf{Postdoctoral researchers supervised:}
\begin{itemize}

\item Calvin McPhail-Snyder, Fall 2021--Fall 2022

\end{itemize}

\smallskip
\noindent\textbf{Graduate students supervised:}
\begin{itemize}

\item Terence Carey (PhD expected: Spring 2028)

\item Logan Gray (PhD expected: Spring 2026) \\
\textbf{Mentored Awards:} 
Linker award, 2025


\item Luke Conners, PhD, Spring 2025 \\
\noindent \textbf{Dissertation:} ``Colored torus link homology'' \\
\noindent \textbf{Placement:} Postdoc at Universit\"{a}t Zurich \\
\textbf{Mentored Awards:}
International Congress of Chinese Mathematicians GTA Silver Award, 2024;
Summer Research Fellowship, 2023 \\
\textbf{Mentored Publications:}
\begin{itemize}
\item L.~Conners,
{\bf Fray functors and equivalence of colored HOMFLYPT homologies}.
2024.
\emph{Preprint}
\href{http://arxiv.org/abs/2405.00875}{\emph{arXiv:2405.00875}}.
79~pages.
\item L.~Conners,
{\bf Row-Column Mirror Symmetry for Colored Torus Knot Homology}.
%2023.
%Submitted for publication. 
%\emph{Preprint}
Selecta Mathematica, 30 (2024), no. 97.
%\href{http://arxiv.org/abs/2303.16271}{\emph{arXiv:2303.16271}}.
86~pages.
\end{itemize}

\item Andrew Adair, PhD, Spring 2024 \\
\noindent \textbf{Dissertation:} ``Categorification of braid group representations'' \\
\noindent \textbf{Placement:} Mathematician at the Department of Defense \\
\textbf{Mentored Awards:} 
Royster Society of Fellows, 2023--2024; 
Dissertation Completion Fellowship, 2023--2024; 
Summer Research Fellowship, 2022

\item Logan Tatham, PhD, Summer 2020 \\
\noindent \textbf{Dissertation:} ``On the quantum type $C$ spider'' \\
\noindent \textbf{Placement:} Mathematician at the Department of Defense \\
\textbf{Mentored Awards:} 
G. de B. Robinson Award of the Canadian Mathematical Society, 2024
\end{itemize}

\smallskip
\noindent\textbf{Undergraduate students supervised:}
\begin{itemize}

\item Gillian Taylor, Honors thesis advisor (defended: March 31, 2022) \\
%and Chancellor's Science Scholar program, Summer 2020--present
\noindent\textbf{Thesis:} ``$3$-manifold invariants in the category $\mathbf{Web}(\mathfrak{sp}_4)$''


\item Abby Watkins, Honors thesis advisor (defended: April 16, 2021) \\
\noindent\textbf{Thesis:} ``The Temperley--Lieb Category and its Trace'' \\
%\noindent Reading supervisor, Fall 2019 and Spring 2020
\noindent \textbf{Placement:} Indiana University PhD program in mathematics

\item Lily Gergle, Honors thesis advisor (defended: April 8, 2021 with Highest Honors) \\
\noindent\textbf{Thesis:} ``Categorification techniques for the Temperley-Lieb category and $\mathbf{Web}(\mathfrak{sp}_4)$'' \\
%\noindent Reading supervisor, Fall 2019 and Spring 2020
\noindent \textbf{Placement:} University of Cambridge (Part III) and University of Illinois Urbana-Champaign \\ 
PhD program in mathematics \\
\textbf{Mentored Awards:} UNC Brauer Prize, 2021

\item Dylan O'Connor, Honors thesis advisor (defended: May 17, 2020) \\
\noindent\textbf{Thesis:} ``An Introduction to the Volume Conjecture'' \\
\noindent \textbf{Placement:} CUNY graduate center PhD program in mathematics

%\item Daniel Cantwell, Undergraduate research advisor, Fall 2016 -- Spring 2017
\end{itemize}


\smallskip
\noindent\textbf{Courses Taught at UNC:}
\begin{itemize}

\item Math 296: Directed Exploration in Mathematics (Monoidal Category Theory), Summer 2025 (1 student)
\item Math 776: Algebraic Topology, Spring 2025 (11 students)
\item Math 381: Discrete Mathematics, Spring 2025 (39 students)
\item Math 891: Categorification in Algebra and Topology, Fall 2024 (8 students)
\item Math 89: First Year Seminar in Knot Theory, Spring 2024 (24 students) %10
\item Math 534: Elements of Modern Algebra, Spring 2024 (25 students)
\item Math 681: Introductory Topology, Fall 2023 (10 students)
\item Math 550: Topology, Fall 2022 (23 students)
\item Math 692H: Honors Thesis in Mathematics, Spring 2022 (1 students)
\item Math 381: Discrete Mathematics, Spring 2022 (39 students)
\item Math 691H: Honors Research in Mathematics, Fall 2021 (1 student)
\item Math 550: Topology, Fall 2021 (28 students)%9
\item Math 231: Calculus of Functions of One Variable I, Fall 2021 (138 students)
\item Math 692H: Honors Thesis in Mathematics, Spring 2021 (2 students)
\item Math 578: Algebraic Structures, Spring 2021 (30 students) %8
\item Math 691H: Honors Research in Mathematics, Fall 2020 (2 students)
\item Math 676: Modules, Linear Algebra, and Groups, Fall 2020 (12 students)
\item Math 381: Discrete Mathematics, Fall 2020 (42 students)
\item Math 296: Directed Exploration in Mathematics (Research on Quantum Invariants), Fall 2020 (1 student)
\item Math 920: Seminar and Directed Readings (Derived Categories), Spring 2020 (2 students)
\item Math 692H: Honors Thesis in Mathematics, Spring 2020 (1 student)
\item Math 681: Introductory Topology, Spring 2020 (11 students) %7
\item Math 534: Elements of Modern Algebra, Spring 2020 (40 students)
\item Math 296: Directed Exploration in Mathematics (Category Theory), Spring 2020 (2 students)
\item Math 920: Seminar and Directed Readings (Homological Algebra), Fall 2019 (2 students)
\item Math 676: Modules, Linear Algebra, and Groups, Fall 2019 (15 students) %6
\item Math 691H: Honors Research in Mathematics, Fall 2019 (1 student)
\item Math 381: Discrete Mathematics, Summer 2019 (9 students)
\item Math 776: Algebraic Topology, Spring 2019 (10 students) %5
\item Math 381: Discrete Mathematics, Spring 2019 (42 students) %4
%\item MATH 994: Doctoral Research and Dissertation, Fall 2019 (1 student)
\item Math 891: Categorification in algebra and topology, Spring 2018 (8 students) %3
%\item MATH 994: Doctoral Research and Dissertation, Spring 2018 (1 student)
\item Math 231: Calculus of Functions of One Variable I, Fall 2017 (148 students)
%\item MATH 994: Doctoral Research and Dissertation, Fall 2017 (2 student)
\item Math 534: Elements of Modern Algebra, Spring 2017 (40 students) %2
\item Math 920: Seminar and Directed Readings (Low-dimensional Topology), Spring 2017 (1 student)
\item Math 231: Calculus of Functions of One Variable I, Fall 2016 (136 students) %1
\end{itemize}

\smallskip
%\newpage
\noindent\textbf{Courses Taught at USC:}
\begin{itemize}
\item Math 245: Mathematics of Physics and Engineering I, Fall 2015
\item Math 225: Linear algebra and linear differential equations, Fall 2015
\item Math 225: Linear algebra and linear differential equations, Spring 2015 
\item Math 226: Calculus III, Fall 2014 (2 sections)
\item Math 435: Vector Analysis and Introduction to Differential Geometry, Spring 2014
\item Math 226: Calculus III, Spring 2014
\item Math 440: Topology, Fall 2013
\item Math 225: Linear algebra and linear differential equations, Spring 2013
\item Math 226: Calculus III, Fall 2012 (2 sections)
\end{itemize}

\smallskip
\noindent\textbf{Courses Taught at Duke University:}
\begin{itemize}
\item Math 490: Algebraic methods in knot theory, Spring 2012
\item Math 105L: Laboratory Calculus and Functions I, Spring 2010
\item Math 122: Introductory Calculus II, Fall 2008
\end{itemize}

\smallskip
%\newpage
\noindent {\bf Curriculum design}:
\begin{itemize}
\item I designed the {\it First year seminar in knot theory} (Math 89) at UNC, 
an inquiry-based seminar on elementary knot theory.

\item I designed the graduate course Math 891, {\it Categorification in algebra and topology} at UNC. 
This course served as an introduction to two aspects of the categorification program: link homology theories and higher representation theory, 
focusing on the $\mathfrak{sl}_2$ case.

\item I designed the undergraduate course Math 490, {\it Algebraic methods in knot theory} at Duke University, through the Bass Fellowship
program. This course served as an introduction to knot theory, the Jones polynomial, and Khovanov homology.
\end{itemize}

\smallskip
\noindent {\bf Other teaching activities}:
\begin{itemize}
\item Program Leader for \emph{UNC Math in Stockholm}, 
a study abroad opportunity in which students travel to Sweden to learn Discrete Mathematics in 
a focused and immersed setting, Summer 2018 -- present. 

\item I worked with high school students from Augustus Hawkins High School in South Central Los Angeles during their mathematical game day at USC on May 6, 2014. 
Working with undergraduate students and Aaron Lauda, I devised a game in which the visiting students analyzed prime knots 
with up to seven crossings. During the game, we explored topics in knot theory including alternating knots, reduced diagrams, the Tait conjectures, and the Jones polynomial.

\item During August 2010 and 2011, 
I was selected to lead week-long programs helping incoming Duke mathematics PhD students prepare for their written qualifying exams.
\end{itemize}

%\newpage
%%\bigskip
%% ---------------------------------------------------------------------------
%\noindent
%{\large \sc Grants}
%
%\vspace{-0.1in}
%\noindent
%\line(1,0){450}
%
%\smallskip
%
%\noindent To date, I have been awarded \$546,000 in grant funding:
%
%\begin{itemize}
%\item {\bf NSF CAREER Grant DMS-2144463}: 
%\emph{Link homology -- in type $A$ and beyond},
%PI: David Rose (percent effort: 16.67\%),
%Award Amount \$425,000 (direct cost: \$285,370),
%Dates: 7/1/2022 -- 6/30/2027
%
%\item {\bf Simons Collaboration Grant}: 
%\emph{Research on knot invariants, representation theory, and categorification}, 
%% Grant number 523992
%PI: David Rose (percent effort: 0\%), 
%Award Amount: \$42,000, 
%Direct Cost: \$35,000,
%Dates: 9/1/2017 -- 8/31/2022
%
%\item {\bf UNC Junior Faculty Development Award},
%PI: David Rose (percent effort: 8.88\%), 
%Award Amount: \$10,000, 
%Dates: 1/1/2019 -- 12/31/2019
%
%\item {\bf NSA Young Investigator Grant}: 
%\emph{ Dualities in higher representation theory and low-dimensional topology}, 
%% Grant number H98230-17-1-0211
%PI: David Rose (percent effort at UNC: 8.33\%), 
%Award Amount: \$40,000, 
%Direct Cost (at UNC): \$17,391
%Dates: 1/21/2016 -- 12/31/2017
%
%\item {\bf AMS Simons Travel Grant}, 
%PI: David Rose,
%Award Amount: \$4,000, 
%Dates: 7/1/2015 -- 12/31/2015
%
% \item {\bf Zumberge Fund Individual Grant Award}: \emph{ Categorified quantum groups and knot homology}, 
% PI: David Rose,
% Award Amount: \$25,000, 
% Dates: 7/1/2013 -- 6/30/2014
%\end{itemize}


%\newpage
%\noindent\textbf{Other:}
%\begin{itemize}
%
%\item {\bf NSF CAREER Grant}: 
%\emph{Mathematics at the Interface of Low-dimensional Topology and Representation Theory} 
%(2020 solicitation), 
%PI: David Rose (percent effort: 22.22\%),
%Requested Budget: \$529,065 (direct cost: \$362,153),
%Recommended for Funding
%(with scores of E/V, E/V, E/V, E/V, V, V, V; not funded due to budgetary restrictions)
%
%\item {\bf NSF Algebra/Number Theory Grant}: 
%\emph{The Algebraic Structure of Link Homology} 
%(2020 solicitation), 
%PI: David Rose (percent effort: 16.67\%),
%Requested Budget: \$263,927 (direct cost: \$177,114),
%(Scores of E/V, V, V, G, G, G; not funded)
%
%\item {\bf NSF Topology Grant}: 
%\emph{Link invariants and new structures in representation theory} 
%(2019 solicitation), 
%PI: David Rose (percent effort: 22.22\%),
%Requested Budget: \$249,763 (direct cost: \$165,239),
%Recommended for Funding If Possible 
%(with scores of E, V, V, V, G, G; not funded due to budgetary restrictions)
%
%\item {\bf NSF Topology Grant}: 
%\emph{Homological invariants of links and braids from categorical representation theory} (2018 solicitation), 
%PI: David Rose (percent effort: 16.67\%),
%Requested Budget: \$253,732 (direct cost: \$166,082),
%Recommended for Funding If Possible 
%(with scores of E/V, E/V, V, V, V/G; not funded due to budgetary restrictions)
%
%\item {\bf NSF CAREER Grant}: 
%\emph{Mathematics at the Interface of Low-dimensional Topology and Representation Theory} (2018 solicitation), 
%PI: David Rose (percent effort: 22.22\%),
%Requested Budget: \$649,062 (direct cost: \$437,815),
%Recommended for Funding If Possible 
%(with scores of E, E/V, E/V, V, V, G; not funded due to budgetary restrictions)
%
%\item {\bf NSF Algebra/Number Theory Grant}: 
%\emph{Applications of representation theory to low-dimensional topology, and vice versa} (2017 solicitation), 
%Recommended for Funding If Possible 
%(with scores of E/V, V, V, G, G, G; not funded due to budgetary restrictions)
%
%\item {\bf NSF Topology Grant}: 
%\emph{Traces, dualities, and invariants of knots and links} (2016 solicitation),
%Recommended for Funding If Possible 
%(with scores of V, V, V; not funded due to budgetary restrictions)
%\end{itemize}


% ---------------------------------------------------------------------------





%\newpage
\bigskip
%% ---------------------------------------------------------------------------
\noindent
{\large \sc Professional Service}

\vspace{-0.1in}
\noindent
\line(1,0){450}
\smallskip

\noindent\textbf{Within UNC Chapel Hill:}

\begin{itemize}

\item Organizer for Mathematics Department Exhibitions at UNC Science Expo, Spring 2017 -- present

\item UNC organizer of the Triangle Topology Seminar, Fall 2016 -- Fall 2019, Fall 2024 -- present

\item Founder and program leader of UNC Math in Stockholm program, Summer 2018 -- present

\item UNC Mathematics Department \sout{Diversity} Community and Outreach Committee, Summer 2020 -- present

\item Faculty co\"{o}rganizer for Girls Talk Math, Spring 2025 -- present

\item UNC Mathematics Undergraduate Advising Committee, Fall 2021 -- Spring 2023, Spring 2025 -- present

\item UNC Mathematics Tenure-Track Mentoring Committee (for Daping Weng), 
Fall 2024 -- present

\item Geometry/Topology Comprehensive Exam Committee, 
Summer 2022 -- present, \\
Summer 2020 -- Summer 2021

\item J. Burton Linker Award Selection Committee, Spring 2024 -- present

\item Organizer for UNC Topology Seminar, Fall 2024 -- Spring 2025

\item UNC Chancellor's Science Scholars program
\begin{itemize}
\item Faculty interviewer, 2025
\item Research mentor, 2020--2021
\end{itemize}

\item UNC Mathematics Colloquium Committee, Summer 2021 -- Spring 2023

\item Algebra Comprehensive Exam Committee, 
Summer 2021 -- Summer 2022, \\
Summer 2019 -- Summer 2020

\item UNC Mathematics Tenure-track hiring committee, Fall 2022

\item UNC Computation Medicine Tenure-track hiring committee, Spring 2022

\item Course coordinator for UNC Math 231, Fall 2021

\item UNC Mathematics Advising Policy Committee, Spring 2018 -- Fall 2018

\item Organizer for Topology Reading Group on Equivariant Cohomology, Fall 2017

\item UNC Mathematics Postdoctoral Hiring Committee, Spring 2017

\item UNC Mathematics Calculus Reform Working Group, Spring 2016 -- Fall 2017

\item Triangle Area Graduate Mathematics Conference Job Panel, December 3, 2016

\item UNC Association for Women in Mathematics Academic Job Search Panel, October 26, 2016

\end{itemize}


\noindent\textbf{To discipline:}

\begin{itemize}



\item Referee and/or quick opinion for 
{\it Advances in Mathematics}, 
{\it Algebraic and Geometric Topology}, 
{\it Annales scientifiques de l'\'{E}cole normale sup\'{e}rieure},
{\it Compositio Mathematica},
%{\it Duke Mathematical Journal},
{\it Fundamenta Mathematicae}, 
{\it Geometry and Topology}, 
{\it International Mathematics Research Notices},
{\it Inventiones Mathematicae},
{\it Journal of the London Mathematical Society},
{\it Mathematische Annalen},
{\it Mathematische Zeitschrift},
and {\it Quantum Topology},
July 2013 -- present

\item Co\"{o}rganizer of the American Institute of Mathematics reading group \emph{Fundamentals of Link Homology},
Summer 2021 -- Winter 2021

\item Grant reviewer for Simons Foundation, Spring 2019

\item Organizer for USC Geometry/Topology Seminar, August 2012 -- Spring 2015

\item USC Geometry/Topology qualifying exam committee, Fall 2013 -- Fall 2015

\item Co\"{o}rganizer of the \emph{Workshop on Categorification and Representation Theory},
Los Angeles, CA, October 30--31, 2014

\item Co\"{o}rganizer of the special session \emph{Categorification in representation theory} at the AMS Fall sectional meeting,
Riverside, CA, November 2--3 2013

\item Organizer for USC graduate Topology reading group on Characteristic Classes, Spring 2013

\item Organizer for Duke Mathematics Department Graduate/Faculty Seminar, January 2010 -- December 2011

\end{itemize}






%% ---------------------------------------------------------------------------









\end{document}